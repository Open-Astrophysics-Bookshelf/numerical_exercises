\label{ch:burgers}

\section{Burgers' equation}

The inviscid Burgers' equation is the simplest {\em nonlinear} hyperbolic
equation:
\begin{equation}
u_t + u u_x = 0
\end{equation}
Here $u$ is both the quantity being advected and the speed at which
it is moving.  Recall that for the linear advection equation, we saw
that the solution was constant along lines $x = ut + x_0$, which are
parallel, since $u$ is spatially constant.  For Burgers' equation,
this is no longer the case, and the characteristic lines are now
given by $dx/dt = u$, with $x(0) = x_0$.  Since $u = u(t)$, we cannot
integrate this directly.

\begin{exercise}[Burgers' characteristics]
{To find the lines along which the solution to Burgers' equation is
constant, consider the full change in $u$:
\begin{equation}
du = \left . \frac{\partial u}{\partial t} \right |_x dt +
     \left . \frac{\partial u}{\partial x} \right |_t dx = 0
\end{equation}
Here, we seek $u = \mathrm{constant}$, so we set $du = 0$.  Now,
using the Burgers' equation itself, show that this implies that
the lines along which the solution is constant are:
\begin{equation}
\frac{dx}{dt} = u(x,t)
\end{equation}
}
\end{exercise}

If we take $u_0 = u(t=0)$, then we can look at how the characteristic
behave over a small time interval (before $u(x,t)$ changes
significantly).  Figure~\ref{fig:burgers_char} shows the behavior for
an initial velocity with sinusoidal profile.  We see that after a
short period of time, the characteristics intersect.  At the point,
$(x_s, t_s)$ where they intersect, there is no way to trace backwards
along the characteristics to find a unique initial state.  This
merging of the characteristics in the $x$-$t$ plane is a {\em shock}, and
represents just one way that nonlinear problems can differ from linear
ones.

Another type of wave not present in a linear system is a {\em
  rarefaction}.  Figure~\ref{fig:burgers_char_rare} shows initial
conditions of slower velocity to the left of faster velocity.  We see
that the characteristics diverge in this case, and we will be left
with having to fill in the solution inbetween as some intermediate
state.


% figure created with figures/burgers/characteristics.py
\begin{figure}[t]
\centering
\includegraphics[width=0.6\linewidth]{burgers-characteristics}
\caption[Characteristics for shock initial conditions]
{\label{fig:burgers_char} (top) Initially sinusoidal
velocity distribution (bottom) Approximate characteristic
structure for Burgers' equation, using $u_0 = u(t)$.  Note that
after a short period of time, the characteristics intersect, signaling
the formation of a shock.}
\end{figure}

% figure created with figures/burgers/characteristics.py
\begin{figure}[t]
\centering
\includegraphics[width=0.6\linewidth]{burgers-characteristics-rare}
\caption[Characteristics for rarefaction initial conditions]
        {\label{fig:burgers_char_rare} (top) Initially discontinuous
          velocity distribution with low velocity left of high
          velocity (bottom) Approximate characteristic structure for
          Burgers' equation, using $u_0 = u(t)$.  Note that after a
          short period of time, the characteristics diverge---this is
          a rarefaction.}
\end{figure}



In conservative form, Burgers' equation appears as:
\begin{equation}
u_t + \left [\tfrac{1}{2} u^2 \right ]_x = 0
\end{equation}
The solution of this follows the same methodology as described in
Chapter~\ref{ch:advection}.  The interface states are predicted as to
second order by Taylor expanding in space and time:
\begin{eqnarray}
u^{n+1}_{i+\myhalf,L}
 &=& u^n_i + \frac{\Delta x}{2} \frac{\partial u}{\partial x}
    + \frac{\Delta t}{2} \left . \frac{\partial u}{\partial t} \right |_i
    + \ldots \\
 &=& u^n_i + \frac{\Delta x}{2} \frac{\partial u}{\partial x}
    + \frac{\Delta t}{2} \left . \left (-u_i \frac{\partial u}{\partial x}
         \right ) \right |_i
    + \ldots \\
 &=& u^n_i + \frac{\Delta x}{2}
   \left ( 1 - \frac{\Delta t}{\Delta x} u_i \right )
   \left . \frac{\partial u}{\partial x} \right |_i + \ldots
\end{eqnarray}
The only difference with the linear advection equation is that now
$u_i \Delta t/\Delta x$ varies from zone to zone, whereas with linear
advection, it is the constant $C$.  The slopes are computed using
the same limiters as with linear advection.

The Riemann problem differs from linear advection.  As we saw above,
the characteristic curves can intersect in the $x$-$t$ plane, and it
is not possible to trace backward from time to learn where the flow
originated.  This is the condition for a {\em shock}.

% figure made by figures/burgers/rh.py
\begin{figure}[t]
\centering
\includegraphics[width=4in]{rh}
\caption[Rankine-Hugoniot conditions]{\label{fig:rh} A rightward moving shock in the $x$-$t$
   plane separating two states: $u_l$ and $u_r$.}
\end{figure}

The shock speed is computed through the {\em Rankine-Hugoniot} jump
conditions.  For a scalar equation, these are easy to construct.
We'll follow the method of \cite{leveque:2002}.  Figure~\ref{fig:rh}
shows two states separated by a rightward moving shock in the $x$-$t$
plane.  At time $t^n$, the state in our interval ($x \in [x_l,x_r]$)
is entirely $u_r$.  As time evolves, we imagine our interval $[x_l,x_r]$ moving
vertically upwards in the diagram, and we see that it contains a mix of states $u_l$ and $u_r$.  Finally, at time, $t^{n+1}$ it is entirely
$u_l$.  The shock moves with a speed $S = \Delta x/\Delta t$ in this
figure.  To determine the speed, we integrate our conservation law over
both space and time (and normalize by $\Delta x = x_r - x_l$):
\begin{equation}
\frac{1}{\Delta x} \int_{x_l}^{x_r} dx \int_{t^n}^{t^{n+1}} dt\, u_t =
  - \frac{1}{\Delta x} \int_{x_l}^{x_r} dx \int_{t^n}^{t^{n+1}} dt \left [ f(u) \right ]_x
\end{equation}
Doing the $t$ integral on the left and $x$ integral on the right, we have
\begin{equation}
\frac{1}{\Delta x} \int_{x_l}^{x_r}\left \{ u(t^{n+1}) - u(t^n) \right \} dx =
  - \frac{1}{\Delta x} \int_{t^n}^{t^{n+1}} \left \{ f(u) |_{x=x_r} - f(u) |_{x=x_l} \right \} dt
\end{equation}
Recognizing that at $t=t^n$, $u = u_r$ and at $t=t^{n+1}$, $u = u_l$,
in the left side becomes
\begin{equation}
\frac{1}{\Delta x} \int_{x_l}^{x_r}\left \{ u(t^{n+1}) - u(t^n) \right \} dx = \{ u(t^{n+1}) - u(t^n) \} =  u_l -u_r \enskip .
\end{equation}
For the right side, we see that all along $x=x_l$ the flux is $f =
f(u_l)$ for $t\in [t^n, t^{n+1}]$.  Likewise, all along $x=x_r$, the
flux is $f = f(u_r)$ in the same time interval (see the figure).
Therefore, our expression becomes:
\begin{equation}
(u_l - u_r) = -\frac{\Delta t}{\Delta x} \left [ f(u_r) - f(u_l)\right ]
\end{equation}
and using $S = \Delta x/\Delta t$, we see
\begin{equation}
S = \frac{f(u_r) - f(u_l)}{u_r - u_l}
\end{equation}

For Burgers' equation, substituting in $f(u) = u^2/2$, we get
\begin{equation}
S = \frac{1}{2}(u_l + u_r)
\end{equation}

With the shock speed known, the Riemann problem is straightforward.  If there
is a shock (compression, so $u_l > u_r$) then we compute the shock speed and
check whether the shock is moving to the left or right, and then use the appropriate
state.  If there is no shock, then we can simply use upwinding, as there is no
ambiguity as to how to trace backwards in time to the correct state.
Putting this together, we have:
\begin{eqnarray}
\mathrm{if~} \underset{\text{(shock)}}{u_l > u_r}:&& u_s = \left \{ \begin{array}{cl}
                u_l & \mathrm{if~} S > 0 \\
                u_r & \mathrm{if~} S < 0 \\
                0   & \mathrm{if~} S = 0 \end{array} \right .   \\[1em]
%
\mathrm{otherwise:}&& u_s = \left \{ \begin{array}{clc}
                u_l & \mathrm{if~} u_l > 0 \\
                u_r & \mathrm{if~} u_r < 0 \\
                0   & \mathrm{otherwise} \end{array} \right .
\end{eqnarray}
\MarginPar{other forms of this from incompressible?}

Once the interface states are constructed, the flux is calculated as:
\begin{equation}
F^{n+\myhalf}_{i+\myhalf} = \frac{1}{2} \left (u_{i+\myhalf}^{n+\myhalf} \right )^2
\end{equation}
and the conservative update is
\begin{equation}
u_i^{n+1} = u_i^n + \frac{\Delta t}{\Delta x}
   \left ( F_{i-\myhalf}^{n+\myhalf} - F_{i+\myhalf}^{n+\myhalf} \right )
\end{equation}

The timestep constraint now must consider the most restrictive Courant
condition over all the zones:
\begin{equation}
\Delta t = \min_i \left \{ \Delta x / u_i \right \}
\end{equation}


\begin{figure}[t]
\centering
\includegraphics[width=0.8\linewidth]{fv-burger-rarefaction}
\caption[Rarefaction solution to the inviscid Burgers'
  equation]{\label{fig:burgers-rarefaction} Solution to the inviscid
  Burgers' equation with 256 zones and a Courant number, $C = 0.8$ for
  initial conditions that generate a rarefaction: the left half of the
  domain was initialized with $u = 1$ and the right half with $u = 2$.
  This initial velocity state creates a divergent flow.  The curves
  are shown 0.02~s apart, with the darker grayscale representing later
  in
  time. \\ \hydroexdoit{\href{https://github.com/python-hydro/hydro_examples/blob/master/burgers/burgers.py}{burgers.py}}}
\end{figure}


Figure~\ref{fig:burgers-rarefaction} shows the solution to Burgers'
equation using the 2nd-order piecewise linear method described here,
with the MC limiter.  The initial conditions chosen are all positive
velocity, with a lower velocity to the left of the higher velocity.
As the solution evolves, the state on the right will rush away from
the state on the left, and spread out like a fan.  This is called a
{\em rarefaction wave} or simply a {\em rarefaction}.

\begin{figure}[t]
\centering
\includegraphics[width=0.8\linewidth]{fv-burger-sine}
\caption[Shock solutions to the inviscid Burgers'
  equation]{\label{fig:burgers-shock} Solution to the inviscid
  Burgers' equation with 256 zones and a Courant number, $C = 0.8$.
  The initial conditions here are sinusoidal and the solution quickly
  steepens into a shock.The curves are shown 0.02~s apart, with the
  darker grayscale representing later in
  time. \\ \hydroexdoit{\href{https://github.com/python-hydro/hydro_examples/blob/master/burgers/burgers.py}{burgers.py}}}
\end{figure}

Figure~\ref{fig:burgers-shock} shows the solution to Burgers'
equation with initially sinusoidal data:
\begin{equation}
\label{eq:burgers:shockic}
\renewcommand{\arraystretch}{1.75}
u(x, t=0) = \left \{ \begin{array}{cc}
    1   & x < 1/3 \\
    1 + \frac{1}{2} \sin\left (\frac{2\pi (x - 1/3)}{1/3}\right ) & 1/3 \le x \le 2/3 \\
    1   & x > 2/3
\end{array}
\right .
\renewcommand{\arraystretch}{1.0}
\end{equation}
This is analogous to the case shown in
Figure~\ref{fig:burgers_char}---we see the solution steepen and form a
shock which propagates to the right.  The shock is practically infinitesimally thin
here, since there is no explicitly viscosity to smear it out\footnote{Numerical diffusion,
in the form of truncation error of our method, will smear things a little.}.

\begin{exercise}[Simple Burgers' solver]
{Extend your 1-d finite-volume solver for advection (from
  Exercise~\ref{adv:ex:fv}) to solve Burgers' equation.  You will
  need to change the Riemann solver and use the local velocity in the
  construction of the interface states.  Run the examples shown in
  Figures~\ref{fig:burgers-rarefaction} and \ref{fig:burgers-shock}}.
\end{exercise}

As we'll see shortly, these two types of waves can also appear in the Euler equations
for hydrodynamics.

A final thing to note is that we solved Burgers' equation in conservative form.
For shock solutions, this is essential, since as we noted earlier, that finite-volume
method relates to the integral form of the PDEs, the discontinuity is nicely handled
by the Riemann solver.  We could also imagine differencing Burgers' equation in
non-conservative form, i.e., starting with:
\begin{equation}
u_t + u u_x = 0
\end{equation}
If you did this (for instance, using a finite difference scheme and an
upwind difference for $u_x$), you would find that you get the wrong
shock speed.

\begin{exercise}[Conservative form of Burgers' equation]
{
Using a simple first-order finite-difference method like we described
in Ch.~\ref{ch:advection} for linear advection, difference the
conservative and non-conservation formulations of Burgers' equation
as:
\begin{equation}
\frac{u^{n+1}_i - u^n_i}{\Delta t} =
   -\frac{1}{2} \frac{(u^n_i)^2 - (u^n_{i-1})^2}{\Delta x}
\end{equation}
and
\begin{equation}
  \frac{u_{i}^{n+1} - u_i^n}{\Delta t} = -
      \frac{u_i^n (u_i^n - u_{i-1}^n)}{\Delta x}
\end{equation}
(Note: these discretizations are upwind so long as $u > 0$).

Run these with the shock initial Riemann conditions:
\begin{equation}
\renewcommand{\arraystretch}{1.75}
u(x, t=0) = \left \{ \begin{array}{cc}
    2   & x < 1/2 \\
    1   & x > 1/2
\end{array}
\right .
\renewcommand{\arraystretch}{1.0}
\end{equation}
and measure the shock speed from your solution by simply differencing
the location of the discontinuity at two different times.  Compare to
the analytic solution for a shock for the Riemann problem.}
\end{exercise}

\section{Characteristic tracing}

A concept that will be useful in the next section is {\em
  characteristic tracing}.  The idea is that we only include
information in the interface states if the characteristic that carries
that information is moving toward the interface.  For Burgers equation,
this is simple, since there is only a single characteristic---the velocity.
So for the left state on an interface, we'd only add the change if
the velocity $u_i > 0$ (moving toward the interface $i+\myhalf$):
\begin{equation}
u^{n+1}_{i+\myhalf,L}
 = u^n_i + \frac{\Delta x}{2}
   \left ( 1 - \frac{\Delta t}{\Delta x} \max(0, u_i) \right )
   \left . \frac{\partial u}{\partial x} \right |_i + \ldots
\end{equation}
Notice that the effect of this is to set the interface state simply to
the value given by the piecewise linear reconstruction on the interface
if the wave isn't moving to the interface.

\section{Going further}

\begin{itemize}
\item The equation we've been dealing with here is the {\em inviscid}
  Burgers' equation.  The full Burgers' equation includes viscosity (a
  velocity diffusion):
  \begin{equation}
    u_t + u u_x = \epsilon u_{xx}
  \end{equation}
  To solve this, we need to first learn about techniques for
  diffusion, and then how to solve equations that span multiple PDE
  types.  This will be described in \S
  \ref{ch:multiphysics:sec:adburgers}.

\item Aside from pedagogical interest, Burgers' equation can be used
  as a simple model of traffic flow (where shocks can arise from
  people slamming on the brakes).  Many sources discuss this
  application, including the text by~\cite{leveque:2002}.
\MarginPar{more refs?}

\end{itemize}
