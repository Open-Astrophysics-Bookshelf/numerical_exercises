Incompressible flow represents the zero-Mach number limit of fluid
flow---no compressibility effects are modeled.  We can extend the
ideas of incompressibe flow to allow us to model some compressibility
effects, giving rise to low Mach number methods.

\section{Low Mach divergence constraints}

\section{Extending the incompressible algorithm}

\subsection{Variable-coefficient elliptic equation}

We now need to solve an elliptic equation of the form:
\begin{equation}
\nabla \cdot (\alpha \nabla \phi) = f
\end{equation}

If we denote the discrete divergence and gradient operators as $D$ and $G$,
then our operator will be $L_\alpha \equiv D \alpha G$.  If we wish to
use a cell-centered discretization for $\phi$, then using a standard 
centered-difference for $D$ and $G$ will result in a stencil that reaches
two zones on either side of the current zone.  This can lead to an
odd-even decoupling. \MarginPar{cite the appropriate Bell paper}

\MarginPar{there is a Pao and Colella (or Pao's thesis?) that also 
discusses issues with cell-centered}

Instead, we again use an approximate projection.  We discretize the
variable-coefficient Laplacian as:
\begin{align}
(L_\alpha \phi)_{i,j} = 
 & \frac{\alpha_{i+1/2,j} (\phi_{i+1,j} - \phi_{i,j}) -
        \alpha_{i-1/2,j} (\phi_{i,j} - \phi_{i-1,j})}{\Delta x^2} + \\
 & \frac{\alpha_{i,j+1/2} (\phi_{i,j+1} - \phi_{i,j}) -
        \alpha_{i,j-1/2} (\phi_{i,j} - \phi_{i,j-1})}{\Delta y^2}
\end{align}

\section{Reactive flows}

\section{Atmospheric flows}

