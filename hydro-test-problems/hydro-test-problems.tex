\section{Testing}

There are a large number of standard hydrodynamics test problems that
should be run on any implementation of an algorithm.  These problems
uncover strengths and weaknesses in your choice of algorithm and also
simple coding bugs.

An ideal test problem has an analytic solution that can be compared to
directly.

Convergence of your results is also an important test.

Recall that for a finite-volume discretization, to second-order, the
value of a function evaluated at the cell-center is the same as the
average of that function over the zone.  Usually this means that you
can simply initialize your test problem using cell-center coordinates.
But for problems that are not well aligned with your grid, e.g., a
spherical initial function mapped onto your Cartesian grid, it is
beneficial to try to initialize the average value in the zone.  A
common way to do this is to sub-divide a zone into a number of
sub-zones, initialize each of the sub-zones, and then average the
sub-zones back to the original zone.  \MarginPar{show figure}


\section{Shock tube problems}


These tests are one-dimensional, and as such, they can provide good
tests of symmetry in your code.  Run the shock tube in the $x$, $y$,
and $z$ directions separately and then compare the profiles---they
should be identical (atleast to machine precision).  If not, there may
be a simple indexing bug, or something like that in your code.

Since these tests start out with a discontinuity, they are not the
best tests to use for convergence testing.  Wherever there is an
initiali discontinuity, the limiters will kick in and drop your
method to first-order accurate.


\subsection{Slow moving shock}


\section{Advection}


\section{Sedov}



\section{Gresho vortex}

The Gresho vortex is a vortex in with a stabilizing pressure gradient
designed such that the overall structure is unchanging in time.  It
has a nice feature in that it allows you to set the Mach number of
the flow as one of the parameters.


\section{Odd-even decompiling}


\section{Hydrostatic equilibrium}

