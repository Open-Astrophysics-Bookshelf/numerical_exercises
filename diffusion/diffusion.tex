\label{ch:diffusion}

\section{Diffusion}

\label{sec:diffusion}

Physically, a diffusive process obeys Fick's law---the quantity that is
diffusing, $\phi$, moves from higher to lower concentration at a rate
proportional to the gradient,
\begin{equation}
  F = - k \nabla \phi
\end{equation}
If we think of this as a diffusive flux, then we can write the time-rate-of-change of $\phi$ as a conservation law:
\begin{equation}
  \phi_t + \nabla\cdot F(\phi) = 0
\end{equation}
This gives rise to the diffusion equation:
\begin{equation}
\phi_t = \nabla \cdot (k \nabla \phi)
\end{equation}

Diffusion has a time-dependence like the advection equations we
already saw, but also has a global character like an elliptic problem.
Unlike advection, there is no sense of upwinding in diffusion.  As we
will see, we can cast our differenced equations in a form reminiscent
of an elliptic problem.

In one-dimension, and assuming that the diffusion coefficient, $k$, is
constant, we have
\begin{equation}
\frac{\partial \phi}{\partial t} =
  \frac{\partial }{\partial x}
  \left ( k \frac{\partial \phi}{\partial x} \right )
\end{equation}


The diffusion equation can describe thermal diffusion (for example, as
part of the energy equation in compressible flow), species/mass
diffusion for multi-species flows, or the viscous terms in
incompressible flows.  In this form, the diffusion coefficient (or
conductivity), $k$, can be a function of $x$, or even $\phi$.


We will consider a constant diffusion coefficient as our model problem:
\begin{equation}
\frac{\partial \phi}{\partial t} = k \frac{\partial^2 \phi}{\partial x^2}
\end{equation}
The diffusion equation is the prototypical parabolic PDE.
The basic behavior of the diffusion equation is to take strongly peaked
concentrations of $\phi$ and smooth them out with time.

\section{Explicit differencing}

A nice overview of the discretizations for diffusion and their stability
is given in \cite{richtmyermorton}.
The simplest way to difference this equation is {\em explicit} in time
(i.e., the righthand side depends only on the old state):
\begin{equation}
\label{eq:diff:explicitdiff}
\frac{\phi_i^{n+1} - \phi_i^n}{\Delta t} =
  k \frac{\phi_{i+1}^n - 2\phi_i^n + \phi_{i-1}^n}{\Delta x^2}
\end{equation}
This is second-order accurate in space, but only first order accurate in
time (since the righthand side is not centered in time).

As with the advection equation, when differenced explicitly, there is
a constraint on the timestep required for stability.  Looking at the
growth of a single Fourier mode, $\phi^n_i = A^n e^{i\imag\theta}$ with $\imag =
\sqrt{-1}$, we find:
\begin{equation}
\label{eq:diff:amplitude}
\frac{A^{n+1}}{A^n} = 1 + 2 \frac{k \Delta t}{\Delta x^2} ( \cos\theta - 1)
\end{equation}
Stability requires that $|A^{n+1}/A^n| \le 1$, which can only be true
if $2k\Delta t/\Delta x^2 \le 1$.  Therefore, our timestep
constraint in this case is
\begin{equation}
\label{eq:diff:dt}
\Delta t < \frac{1}{2} \frac{\Delta x^2}{k}
\end{equation}
As with advection, we often write the timestep as
\begin{equation}
\Delta t = \frac{C}{2} \frac{\Delta x^2}{k}
\end{equation}
where $C$ is a constant, $C < 1$.
Note the $\Delta x^2$ dependence---this constraint can become really
restrictive.

\begin{exercise}
[Explicit diffusion stability condition]
Derive Eq.~\ref{eq:diff:amplitude} by inserting $\phi^n_i = A^n e^{ij\theta}$
into Eq.~\ref{eq:diff:explicitdiff}
\end{exercise}

To complete the solution, we need boundary conditions at the left
($x_l$) and right ($x_r$) boundaries.  These are typically either
Dirichlet:
\begin{eqnarray}
\phi |_{x=x_l} &=& \phi_l\\
\phi |_{x=x_r} &=& \phi_r
\end{eqnarray}
or Neumann:
\begin{eqnarray}
\phi_x |_{x=x_l} &=& \phi_x |_l\\
\phi_x |_{x=x_r} &=& \phi_x |_r
\end{eqnarray}
Physically, a Dirichlet BC means fixing the value of $\phi$ (e.g., temperature)
on the boundary.  A Neumann BC means fixing the flux on the boundary.

Like with advection, it helps to have a good test problem to evaluate our
methods for diffusion.  We can use the fact that the diffusion of a Gaussian
is a Gaussian with a smaller amplitude and greaterwidth.

\begin{exercise}[1-d explicit diffusion]
{Write a one-dimensional explicit diffusion solver for the
  domain $[0,1]$ with Neumann boundary conditions at each end and $k = 1$,
  using the discretization of Eq.~\ref{eq:diff:explicitdiff}.

  If we begin with a Gaussian, the resulting solution is also a Gaussian,
  giving a solution\footnote{Note: the 2- and 3-d solutions are slightly different than this 1-d solution}:
  \begin{equation}
   \phi(x,t) = (\phi_2 - \phi_1) \sqrt{\frac{t_0}{t + t_0}} e^{-\frac{1}{4}(x - x_c)^2/k(t+t_0)} + \phi_1
  \end{equation}
  Initialize our problem with $t = 0$, and take
  $t_0 = 0.001$, $\phi_1 = 1$, and $\phi_2 = 2$, and $x_c$ is the
  coordinate of the center of the domain.  Run until $t = 0.01$ and
  compare to the analytic solution above.}
\end{exercise}


Figure~\ref{fig:diff_explicit} shows the solution for 64 zones
and a value of $C = 0.8$.  We see good agreement with the
analytic solution.  Note that if the initial conditions are
not well resolve initially, then the solution will be bad
(see Figure~\ref{fig:diff_explicit_res}).


\begin{figure}
\centering
\includegraphics[width=\linewidth]{diff-explicit-64}
\caption[Explicit diffusion of a Gaussian]{\label{fig:diff_explicit}
Diffusion of a Gaussian using the explicit differencing of
Eq.~\ref{eq:diff:explicitdiff} with 64 zones and C = 0.8, shown at
several times.  The dotted line is the analytic
solution.  \\ \hydroexdoit{\href{https://github.com/python-hydro/hydro_examples/blob/master/diffusion/diffusion_explicit.py}{diffusion\_explicit.py}}}
\end{figure}


\begin{figure}
\centering
\includegraphics[width=\linewidth]{diffexplicit-res}
\caption[Underresolved explicit diffusion of a Gaussian]{\label{fig:diff_explicit_res}
Diffusion of a Gaussian using the explicit differencing of
Eq.~\ref{eq:diff:explicitdiff} with different resolutions.
several times.  The dotted line is the analytic
solution.  \\ \hydroexdoit{\href{https://github.com/python-hydro/hydro_examples/blob/master/diffusion/diffusion_explicit.py}{diffusion\_explicit.py}}}
\end{figure}

As with advection, if we exceed the timestep limit ($C > 1$), then
the solution is unstable.  This is shown in Figure~\ref{fig:diff_explicit_bad}.

\begin{figure}
\centering
\includegraphics[width=\linewidth]{diff-explicit-64-bad}
\caption[Unstable explicit diffusion]{\label{fig:diff_explicit_bad}
Diffusion of a Gaussian using the explicit differencing of
Eq.~\ref{eq:diff:explicitdiff} with 64 zones, but a timestep
with $C > 1$, showing that the solution is unstable.
 \\ \hydroexdoit{\href{https://github.com/python-hydro/hydro_examples/blob/master/diffusion/diffusion_explicit.py}{diffusion\_explicit.py}}}
\end{figure}

Our spatial order-of-accuracy is second-order.
Figure~\ref{fig:diffexplicit_converge} shows the error as a function
of number of zones, using the $L_2$ norm of the solution with respect
to the analytic solution.  Notice that at the coarsest resolution the
error is very high---we are not resolving the initial conditions well
enough to have a meaningful solution.  At higher resolutions, we
converge as $O(\Delta x^2)$.  Recall though that our method has a
truncation error that is $O(\Delta t) + O(\Delta x^2)$, but we don't
see the first-order in time scaling.  This is because of the timestep
restriction.  Since $\Delta t \sim \Delta x^2/k$, as we cut $\Delta x$
by 2, $\Delta t$ drops by 4, so the timestep choice makes our
$O(\Delta t)$ truncation error go as $O(\Delta x^2)$.

Note, this is not the case for advection, where $\Delta t \sim \Delta x$, so
for our advection discretizations we will want the same order of accuracy
in our spatial and temporal discretizations.

\begin{figure}
\centering
\includegraphics[width=\linewidth]{diffexplicit-converge-0_8}
\caption[Error convergence of explicit diffusion]{\label{fig:diffexplicit_converge}
Error convergence with resolution of the explicit diffusion using the differencing of
Eq.~\ref{eq:diff:explicitdiff}
 \\ \hydroexdoit{\href{https://github.com/python-hydro/hydro_examples/blob/master/diffusion/diffusion_explicit.py}{diffusion\_explicit.py}}}
\end{figure}

\section{Implicit with direct solve}

Recall that an implicit discretization of advection did not have a
timestep restriction for stability.  The same holds true for
diffusion.  A backward-Euler implicit discretization would be:
\begin{equation}
\label{eq:diff:foimplicit}
\frac{\phi_i^{n+1} - \phi_i^n}{\Delta t} =
  k \frac{\phi_{i+1}^{n+1} - 2\phi_i^{n+1} + \phi_{i-1}^{n+1}}{\Delta x^2}
\end{equation}
The only difference with Eq.~\ref{eq:diff:explicitdiff} is the
time-level of $\phi$ on the righthand side.  Defining:
\begin{equation}
\alpha \equiv k \frac{\Delta t}{\Delta x^2}
\end{equation}
This is still first-order in time.  We can do the stability analysis
to see the growth of a mode, giving,
\begin{equation}
\frac{A^{n+1}}{A^n} = \frac{1}{1 - 2\alpha (\cos\theta - 1)}
\end{equation}
We see that $|A^{n+1}/A^n| \le 1$ for all $\theta$, $\alpha$, so this
discretization is unconditionally stable.  However, the timestep will
still determine the accuracy.

We can write Eq.~\ref{eq:diff:foimplicit} as:
\begin{equation}
\label{eq:diff:implicit}
-\alpha \phi_{i+1}^{n+1} + (1 + 2\alpha) \phi_{i}^{n+1} - \alpha \phi_{i-1}^{n+1} = \phi_i^n
\end{equation}
This is a set of coupled algebraic equations.  We can write this in
matrix form.  Using a cell-centered grid, we will solve for the values
$[\mathrm{lo},\mathrm{hi}]$.
The implicit method can use any $C > 0$.

We specify boundary conditions by modifying the stencil
(Eq.~\ref{eq:diff:implicit}) for the updates to $\mathrm{lo}$ and
$\mathrm{hi}$\footnote{Here, we use the $\mathrm{lo}$, $\mathrm{hi}$
  notation for grid indices from \S~\ref{sec:fv:bcs}}.  For example, homogeneous
Neumann BCs on the left mean:
\begin{equation}
\phi_\mathrm{lo-1} = \phi_\mathrm{lo}
\end{equation}
and substituting this into Eq~\ref{eq:diff:implicit}, the update for the leftmost cell is:
\begin{equation}
 (1 + \alpha) \phi_\mathrm{lo}^{n+1} -\alpha \phi_\mathrm{lo+1}^{n+1}  =
  \phi_\mathrm{lo}^n
\end{equation}
If we choose Dirichlet BCs on the right ($\phi |_{x=x_l} = A$), then:
\begin{equation}
\phi_\mathrm{hi+1} = 2 A - \phi_\mathrm{hi}
\end{equation}
Substituting this into Eq~\ref{eq:diff:implicit} the update for the rightmost cell is:
\begin{equation}
- \alpha \phi_\mathrm{hi-1}^{n+1} + (1 + 3\alpha) \phi_\mathrm{hi}^{n+1}  =
  \phi_\mathrm{hi}^n + \alpha 2 A
\end{equation}
For all other interior cells, the stencil is unchanged.  The resulting
system appears as a {\em tridiagonal} matrix.
\begin{equation}
\renewcommand\arraystretch{1.5}
\left (
\begin{array}{ccccccc}
1+\alpha &   -\alpha &           &        &         &           &          \\
-\alpha  & 1+2\alpha & -\alpha   &        &         &           &          \\
         & -\alpha   & 1+2\alpha & -\alpha&         &           &          \\
         &           & \ddots    & \ddots & \ddots  &           &          \\
         &           &           & \ddots & \ddots  & \ddots    &          \\
         &           &           &        & -\alpha & 1+2\alpha &-\alpha   \\
         &           &           &        &         & -\alpha   &1+3\alpha \\
\end{array}
\right )
\left (
\begin{array}{c}
\phi_\mathrm{lo}^{n+1} \\
\phi_\mathrm{lo+1}^{n+1} \\
\phi_\mathrm{lo+2}^{n+1} \\
\vdots \\
\vdots \\
\phi_\mathrm{hi-1}^{n+1} \\
\phi_\mathrm{hi}^{n+1} \\
\end{array}
\right )
=
\left (
\begin{array}{c}
\phi_\mathrm{lo}^{n} \\
\phi_\mathrm{lo+1}^{n} \\
\phi_\mathrm{lo+2}^{n} \\
\vdots \\
\vdots \\
\phi_\mathrm{hi-1}^{n} \\
\phi_\mathrm{hi}^{n} + \alpha 2 A\\
\end{array}
\right )
\end{equation}
This can be solved by standard matrix operations, using a tridiagonal
solvers (for example).  Notice that the ghost cells do not appear in this
linear system---we are only updating the interior points.

\subsubsection{Crank-Nicolson time discretization}

A second-order in time discretization requires us to center the
righthand side in time.  We do this as:
\begin{equation}
\frac{\phi_i^{n+1} - \phi_i^n}{\Delta t} =
  \frac{k}{2} \left ( \frac{\phi_{i+1}^{n} - 2\phi_i^{n} + \phi_{i-1}^{n}}{\Delta x^2} +
                      \frac{\phi_{i+1}^{n+1} - 2\phi_i^{n+1} + \phi_{i-1}^{n+1}}{\Delta x^2} \right )
\end{equation}
This looks like the average of the explicit and implicit systems we just saw.
This time-discretization is called {\em Crank-Nicolson}.  Again, using
$\alpha \equiv k\Delta t / \Delta x^2$, and grouping all the $n+1$
terms on the left we have:
\begin{equation}
\phi^{n+1}_i - \frac{\alpha}{2} \left ( \phi^{n+1}_{i+1} - 2\phi^{n+1}_i + \phi^{n+1}_{i-1} \right )
  = \phi^n_i + \frac{\alpha}{2} \left ( \phi^{n}_{i+1} - 2\phi^{n}_i + \phi^{n}_{i-1} \right )
\end{equation}
and grouping together the the $n+1$ terms by zone, we have:
\begin{equation}
-\frac{\alpha}{2} \phi^{n+1}_{i+1} + (1 + \alpha)\phi^{n+1}_i - \frac{\alpha}{2} \phi^{n+1}_{i-1}
  = \phi^n_i + \frac{\alpha}{2} \left ( \phi^n_{i+1} - 2 \phi^n_i + \phi^n_{i-1} \right )
\end{equation}

Considering homogeneous Neumann boundary conditions on the left and
right, we again have $\phi^{n+1}_\mathrm{lo-1} =
\phi^{n+1}_\mathrm{lo}$ and $\phi^{n+1}_\mathrm{hi+1} =
\phi^{n+1}_\mathrm{hi}$, and our stencil at the boundary becomes
\begin{equation}
-\frac{\alpha}{2} \phi^{n+1}_\mathrm{lo+1} + \left (1 + \frac{\alpha}{2} \right ) \phi^{n+1}_\mathrm{lo} =
    \phi^n_\mathrm{lo} + \frac{\alpha}{2} \left ( \phi^{n}_\mathrm{lo+1} - 2\phi^{n}_\mathrm{lo} + \phi^{n}_\mathrm{lo-1} \right )
\end{equation}

The matrix form of this system is:
\newcommand{\atwo}{\frac{\alpha}{2}}
\begin{equation}
\label{eq:diff:cnmatrix}
\renewcommand\arraystretch{1.5}
\left (
\begin{array}{ccccccc}
1+\atwo &   -\atwo &          &        &        &           &          \\
-\atwo  & 1+\alpha & -\atwo   &        &        &           &          \\
        &   -\atwo & 1+\alpha & -\atwo &        &           &          \\
        &          & \ddots   & \ddots & \ddots &           &          \\
        &          &          & \ddots & \ddots & \ddots    &          \\
        &          &          &        & -\atwo & 1+\alpha  &-\atwo   \\
        &          &          &        &        & -\atwo    &1+\atwo \\
\end{array}
\right )
\left (
\begin{array}{c}
\phi_\mathrm{lo}^{n+1} \\
\phi_\mathrm{lo+1}^{n+1} \\
\phi_\mathrm{lo+2}^{n+1} \\
\vdots \\
\vdots \\
\phi_\mathrm{hi-1}^{n+1} \\
\phi_\mathrm{hi}^{n+1} \\
\end{array}
\right )
=
\left (
\begin{array}{c}
\phi_\mathrm{lo}^{n}  + \frac{k\Delta t}{2} [\nabla^2 \phi]^n_\mathrm{lo}\\
\phi_\mathrm{lo+1}^{n} + \frac{k\Delta t}{2} [\nabla^2 \phi]^n_\mathrm{lo+1} \\
\phi_\mathrm{lo+2}^{n} + \frac{k\Delta t}{2} [\nabla^2 \phi]^n_\mathrm{lo+2}\\
\vdots \\
\vdots \\
\phi_\mathrm{hi-1}^{n} + \frac{k\Delta t}{2} [\nabla^2 \phi]^n_\mathrm{hi-1}\\
\phi_\mathrm{hi}^{n} + \frac{k\Delta t}{2} [\nabla^2 \phi]^n_\mathrm{hi}\\
\end{array}
\right )
\end{equation}

Figure~\ref{fig:diffuse} shows the result of using $\alpha = 0.8$ and $\alpha = 8.0$.  We
see that they are both stable, but that the smaller timestep is closer to the analytic
solution (especially at early times).

\begin{figure}[t]
\centering
\includegraphics[width=0.75\linewidth]{diff-implicit-128-CFL_0_8}\\
\includegraphics[width=0.75\linewidth]{diff-implicit-128-CFL_8_0}
\caption[Implicit diffusion of a Gaussian]{\label{fig:diffuse}
  Implicit diffusion of a Gaussian (with Crank-Nicolson
  discretization) with $C = 0.8$ and $C = 8.0$.  The exact solution at
  each time is shown as the dotted
  line. \\ \hydroexdoit{\href{https://github.com/python-hydro/hydro_examples/blob/master/diffusion/diffusion_implicit.py}{diffusion\_implicit.py}}}
\end{figure}

\begin{exercise}[1-d implicit diffusion]
{Write a one-dimensional implicit diffusion solver for the
  domain $[0,1]$ with Neumann boundary conditions at each end and $k = 1$.
  Your solver should use a tridiagonal solver and initialize a matrix like
  that above.  Use a timestep close to the explicit step, a grid with
  N = 128 zones.

  Use a Gaussian for your initial conditions (as you did for the
  explicit problem.}
\end{exercise}


\section{Implicit multi-dimensional diffusion via multigrid}
\label{diff:sec:implicit_mg}

Instead of doing a direct solve of the matrix form of the system, we
can use multigrid techniques.  Consider the Crank-Nicolson system we just
looked at:
\begin{equation}
\frac{\phi^{n+1}_i - \phi^n_i}{\Delta t} =
   \frac{1}{2} \left ( k \nabla^2 \phi^n_i + k \nabla^2 \phi^{n+1}_i \right )
\end{equation}
Grouping all the
$n+1$ terms on the left, we find:
\begin{equation}
\phi^{n+1}_i - \frac{\Delta t}{2} k \nabla^2 \phi^{n+1}_i =
    \phi^n_i + \frac{\Delta t}{2} k \nabla^2 \phi^n_i
\end{equation}
This is in the form of a constant-coefficient Helmholtz equation,
\begin{equation}
\label{eq:diff:helmholtz}
(\alpha - \beta \nabla^2) \phi^{n+1} = f
\end{equation}
with
\begin{eqnarray}
\alpha &=& 1 \\
\beta &=& \frac{\Delta t}{2} k \\
f &=& \phi^n + \frac{\Delta t}{2} k \nabla^2 \phi^n
\end{eqnarray}
This can be solved using multigrid techniques with a Helmholtz
operator.  The same boundary conditions discussed in
Chapter~\ref{ch:multigrid} apply here.  The main difference between
the multigrid technique for the Poisson problem and the Helmholtz
problem is the form of the smoother.  In 1-d, we discretize
Eq.~\ref{eq:diff:helmholtz} with a second-order difference expression
for the second derivative, and isolate $\phi_i$, giving a smoothing
operation of the form:
\begin{equation}
\label{eq:diff:discrete1d}
\phi_{i} \leftarrow
 \left .    \left ( f_{i} + \frac{\beta}{\Delta x^2} [\phi_{i+1}
                             + \phi_{i-1}] \right ) \middle /
\left ( \alpha + \frac{2 \beta}{\Delta x^2}  \right )  \right .
\end{equation}
Note that when you take $\alpha = 0$ and $\beta = -1$ in
Eq.~\ref{eq:diff:discrete1d}, the smoothing operation reduces to the
form that we saw in Chapter~\ref{ch:multigrid} for just the Poisson
equation.

Recall, when using multigrid, you do not need to actually construct the
matrix.  This is usually the most efficient way to implement diffusion
in a multi-dimensional simulation code, especially when distributing
the grid across parallel processors.  Since the discretization is the
same as the direct matrix solve, Eq.~\ref{eq:diff:cnmatrix}, the
result will be exactly the same (to the tolerance of the multigrid
solver).

\begin{figure}[t]
\centering
\includegraphics[width=0.48\linewidth]{gauss_diff_start}
\includegraphics[width=0.48\linewidth]{gauss_diff_end}
\caption[2-d diffusion of a Gaussian]{\label{fig:diff:twodcompare}
  Diffusion of a Gaussian in 2-d with $128^2$ zones using $k = 1.0$
  and $\cfl = 2.0$.  This used Crank-Nicolson time-discretization and
  multigrid.  This can be run in \pyro\ as {\tt ./pyro diffusion
    gaussian inputs.diffusion}.}
\end{figure}

\begin{figure}[t]
\centering
\includegraphics[width=0.8\linewidth]{gauss_diffusion_compare}
\caption[Comparison of 2-d implicit diffusion with analytic
  solution]{\label{fig:diffusion:multid} A comparison of 2-d implicit
  diffusion from Figure~\ref{fig:diff:twodcompare} with the analytic solution at several times.
  The 2-d solution was averaged over angles to yield a profile as a
  function of radius, using the {\tt
    gauss\_diffusion\_compare.py} in \pyro.}
\end{figure}

Figure~\ref{fig:diffusion:multid} shows the result of diffusing a
Gaussian profile on a 2-d grid using Crank-Nicolson
time-discretization, a constant coefficient, and multigrid to solve
the discretized system, using \pyro.  There is a good agreement
between the analytic and numerical solution, showing that this scheme
models the diffusion of an initially resolved Gaussian well.


\begin{exercise}[Implicit multi-dimensional diffusion]
{The diffusion solver in \pyro\ uses Crank-Nicolson differencing in time.
Modify the solver to do first-order backward Euler.  This will change
the form of the linear system (coefficients and righthand side), but
should not require any changes to the multigrid solver itself.

Compare the solution with the Crank-Nicolson solution for a very coarsely-resolved
Gaussian and a finely-resolved Gaussian.
}
\end{exercise}


\subsection{Convergence}

One needs to be careful with the Crank-Nicolson discretization.  If
the initial data is under-resolved and you are taking a big timestep
($C \gg 1$), then the method can be unstable.
Figure~\ref{fig:diff:cnunstable} shows such a case for the Gaussian
diffusion with Crank-Nicolson discretization and 64 zones with $C =
10$.  For this reason, simulation codes often drop down to the
simple backwards difference time-discretization for implicit diffusion
of under-resolved flows.  A good discussion of the different types
of stability can be found in \cite{leveque:fd}.

\begin{figure}
\centering
\includegraphics[width=\linewidth]{diff-implicit-64-CFL_10_0}
\caption[Under-resolved Crank-Nicolson diffusion]
{\label{fig:diff:cnunstable} Crank-Nicolson diffusion of a Gaussian
on under-resolved initial data with a large timestep.  Here we use
$64$ zones and $C= 10$.  \\ \hydroexdoit{\href{https://github.com/python-hydro/hydro_examples/blob/master/diffusion/diffusion_implicit.py}{diffusion\_implicit.py}}}
\end{figure}


\begin{figure}
\centering
\includegraphics[width=0.75\linewidth]{diffimplicit-converge-0_8} \\
\includegraphics[width=0.75\linewidth]{diffimplicit-converge-8_0}
\caption[Convergence of diffusion methods]
{\label{fig:diff:convergence} Convergence of the explicit, backward-difference,
and Crank-Nicolson diffusion methods for $C = 0.8$ (top) and $C = 8.0$ (bottom).
For the latter case, the explicit method is not valid and is not shown.
We see that the Crank-Nicolson method has the lowest error, and when resolving
the data, has second-order convergence.\\
\hydroexdoit{\href{https://github.com/python-hydro/hydro_examples/blob/master/diffusion/diff_converge.py}{diff\_converge.py}}}
\end{figure}


Figure~\ref{fig:diff:convergence} shows the convergence of several of
the methods discussed here on the diffusion of a Gaussian, for two
different Courant numbers, 0.8 and 8.0.  For the lower case, we can do
the diffusion explicitly as well as implicitly, and we see that all
methods show the same trends, with the Crank-Nicolson method being the
most accurate (since it is the only second-order-in-time method
shown).  For the larger timestep, we only run the implicit methods
(the explicit case is unstable).  We see that at coarse resolution,
the errors in the methods are similar---this is because the solution
is unresolved.  At higher resolution, the error for the Crank-Nicolson
method is an order of magnitude lower.



\section{Non-constant Conductivity}

For a non-constant conductivity, our equation has the form:
\begin{equation}
\frac{\partial \phi}{\partial t} = \frac{\partial}{\partial x} \left (k \frac{\partial \phi}{\partial x} \right )
\end{equation}

For the case where $k = k(x)$,
  we discretize as:
  \begin{equation}
  \frac{\phi_i^{n+1} - \phi_i^n}{\Delta t} =
        \frac{ \{ k \nabla \phi \}_{i+\myhalf} -
               \{ k \nabla \phi \}_{i-\myhalf}}{\Delta x}
  \end{equation}
 Here we need the values of $k$ at the interfaces, $k_{i-\myhalf}$ and
 $k_{i+\myhalf}$.  We can get these from the cell-centered values in a
 variety of ways including straight-averaging:
 \begin{equation}
 k_{i+\myhalf} = \frac{1}{2} (k_i + k_{i+1})
 \end{equation}
 or averaging the inverses:
 \begin{equation}
 \frac{1}{k_{i+\myhalf}} = \frac{1}{2} \left (\frac{1}{k_i} + \frac{1}{k_{i+1}} \right )
 \end{equation}
The actual form should be motivated by the physics

Slightly more complicated are state-dependent transport
coefficients---the transport coefficients themselves depend on the
quantity being diffused:
  \begin{equation}
  \frac{\phi_i^{n+1} - \phi_i^n}{\Delta t} =
        \frac{1}{2} \left \{
               \nabla \cdot [ k(\phi^n) \nabla \phi^n ]_i +
               \nabla \cdot [ k(\phi^{n+1}) \nabla \phi^{n+1} ]_i
               \right \}
  \end{equation}
  (for example, with thermal diffusion, the conductivity can
  be temperature dependent).  In this case, we can achieve second-order
  accuracy by doing a predictor-corrector.  First we diffuse with
  the transport coefficients evaluated at the old time, giving a provisional
  state, $\phi^\star$:
  \begin{equation}
  \frac{\phi_i^\star - \phi_i^n}{\Delta t} =
        \frac{1}{2} \left \{
               \nabla \cdot [ k(\phi^n) \nabla \phi^n ] +
               \nabla \cdot [ k(\phi^n) \nabla \phi^\star ]
               \right \}
  \end{equation}
  Then we redo the diffusion, evaluating $k$ with $\phi^\star$ to
  center the righthand side in time, giving the new state, $\phi^{n+1}$:
  \begin{equation}
  \frac{\phi_i^{n+1} - \phi_i^n}{\Delta t} =
        \frac{1}{2} \left \{
               \nabla \cdot [ k(\phi^n) \nabla \phi^n ] +
               \nabla \cdot [ k(\phi^\star) \nabla \phi^{n+1} ]
               \right \}
  \end{equation}
  This is the approach used, for example, in \cite{SNpaper}.


\section{Diffusion in Hydrodynamics}

Often we find diffusion represented as one of many physical processes
in a single equation.  For example, consider the internal energy
equation with both reactions and diffusion:
 \begin{equation}
 \rho\frac{\partial e}{\partial t} + \rho U \cdot \nabla e + p \nabla \cdot U = \nabla \cdot k \nabla T + \rho S
 \end{equation}
 This can be solved via an explicit-implicit discretization.  First
 the advection terms are computed as:
 \begin{equation}
 A = \rho U \cdot \nabla e + p \nabla \cdot U
 \end{equation}
 Then the advective-diffusive part is solved implicitly.  Expressing
 $e = e(\rho, T)$, and using the chain rule,
 \begin{equation}
   \nabla e = e_T \nabla T + e_\rho \nabla \rho
 \end{equation}
 where $e_T = \partial e / \partial T |_\rho \equiv c_v$ is the
 specific heat at constant volume and $e_\rho = \partial e / \partial
 \rho |_T$.  Rewriting, we have:
 \begin{equation}
 \nabla T = (\nabla e - e_\rho \nabla \rho)/c_v
 \end{equation}
  and then
 \begin{equation}
 \rho \frac{\partial e}{\partial t} = \nabla \cdot (k/c_v) \nabla e - \nabla \cdot (k e_\rho/c_v) \nabla \rho -A + \rho S
 \end{equation}
 This is now a diffusion equation for $e$, which can be solved by the
 techniques described above.  Note: if the transport coefficients
 (including $c_v$, $e_\rho$) are dependent on $e$, then we still need
 to do a predictor-corrector method here.  This is discussed, for
 example, in \cite{SNpaper,malone:2011}.  A simple case for this type of
 advection-diffusion is also shown in \S~\ref{sec:multiphys:diffreact}.
